\documentclass[10pt, aspectratio=169]{beamer}

% Makes sure we can access the aquila theme files
\makeatletter
\def\input@path{{../}}
\makeatother

\usepackage{xspace}
\usepackage{dsfont}
\usepackage{amsmath}
\usepackage{graphicx}
\usepackage{booktabs}
\usepackage{svg}
\usepackage{xspace}
\usepackage{fontawesome}

\renewcommand{\checkmark}{\faCheck}
\newcommand{\xmark}{\faClose}

%%% Math macros
\newcommand{\ive}[1]{\llbracket#1\rrbracket}

\newenvironment{rcases}{\left.\begin{aligned}}{\end{aligned}\right\rbrace}

\newcommand{\expectation}[2]{\ensuremath{\mathbb{E}_{#1}\left[#2\right]}\xspace}
\newcommand{\prob}[1]{\mathbb{P}\left(#1\right)}
\newcommand{\bools}{\ensuremath{\mathbb{B}}\xspace}
\newcommand{\fuzzy}{\ensuremath{\mathbb{F}}\xspace}
\newcommand{\diff}{\ensuremath{\ \mathrm{d}}}
\newcommand{\neuralparams}{\ensuremath{\boldsymbol{\Lambda}}\xspace}
\newcommand{\params}{\ensuremath{\boldsymbol{\theta}}\xspace}
\renewcommand{\vector}[1]{\ensuremath{\boldsymbol{#1}}\xspace}


\usetheme{aquila}

\titlegraphic{}
\title{On the principles of making a clear presentation}
\subtitle{based on the work of Jean-luc Doumont}
\author{Lennert De Smet}

\begin{document}

\maketitle

\begin{frame}

\attentioncenter{
    You start your presentation with an attention-grabbing statement, to capture the interest of your audience
}
    
\end{frame}

\begin{frame}
    
\frametitle{The introduction needs to answer the questions \\ of what, why and how}

\goldencolumn{
\shift{need}
}{
\emph{What is the setting that you are going to discuss?} \\
This part is for the audience and is \accent{not} about you yet
}

\step

\goldencolumn{
\shift{task}
}{
\emph{Why is this setting interesting?} \\
This part explains the motivating \accent{research gap}
}

\step

\goldencolumn{
\shift{message}
}{
\emph{How are you solving the problem?} \\
Now you talk about \accent{you} and your main message
}

\end{frame}


\begin{frame}

\begin{enumerate}
    \only<1>{
    \item You can use a table of contents to structure your presentation in \accent{3 - 5} points
    }
    \only<2>{
    \item \textbf{Feel free to highlight and introduce the current point that you are going to discuss}
    }
    \babystep
    \item The table of content serves as a \accent{preview} and helps the audience to follow
    \babystep
    \item While optional, the audience should always feel like they know where they are
\end{enumerate}
    
\end{frame}

\begin{frame}
\frametitle{The meat of your presentation starts now \\ by discussing your first point}

\goldencolumn{}{
\emph{A single point does not have to be limited} \\
to a single slide

\step

\emph{Longer presentations will cover each point} \\
over multiple subsequent slides

\step

\emph{Focus on keeping a clear structure} \\
and \accent{always} distinctively transition between points
}
    
\end{frame}

\begin{frame}
\frametitle{The title is a full sentence conveying the main message \\ using meaningful line breaks}

\goldencolumn{}{
\emph{A shifted \command{goldencolumn} contains the main text} \\
that is of interest to the current slide

\step

\emph{Each sentence is consistently and cleanly aligned} \\
and contains only the most necessary information

\step

\emph{Sentences are best limited to two lines} \\
to maintain a high signal-to-noise ratio
}

\end{frame}

\begin{frame}

\begin{enumerate}
    \item You can use a table of contents to structure your presentation in \accent{3 - 5} points
    \babystep
    \only<1>{
    \item The table of content now serves as an \accent{anchor} and helps the audience to follow
    }
    \only<2>{
    \item \textbf{Take this time to link the previous point to the next one}
    }
    \babystep
    \item While optional, the audience should always feel like they know where they are
\end{enumerate}
    
\end{frame}

\begin{frame}
\frametitle{Consistency is key for a clear presentation \\ but you can change things up}

\begin{center}
    A crucial statement or equation can deserve a more central presentation
\end{center}
{\LARGE
\begin{align*}
    G_{\mu \nu} + \Lambda g_{\mu \nu} = 
    \colournode{orange_red}{
        $\displaystyle\frac{8\pi G}{c^4} T_{\mu \nu}$
        % $\kappa T_{\mu \nu}$
    }{crucial_term}{0.1em}
\end{align*}
}

\vspace{1.5em}

\goldencolumn{}{
\emph{
    You can visually link to \colournode{white}{
        \colouraccent{orange_red}{crucial parts}
    }{crucial_part}{0.1em}
} \\
by using colour and connections
}

% A little bit of tikz never hurt anyone, right?
\begin{tikzpicture}[overlay, remember picture]
\node[between=crucial_term.south and crucial_part.north] (midpoint) {};

\draw[thinline, shorten <= 1mm, shorten >= 1mm, draw=yellow_orange] (crucial_term.south) -- (crucial_term.south |- midpoint) -| (crucial_part.north);

\end{tikzpicture}

\reference{
    Einstein, Albert.
    (1916)
}

\end{frame}

\begin{frame}
\frametitle{Figures can always be shown and discussed \\ because of the refreshing amount of whitespace}

\goldencolumn{
    \pictureshift
    \roundpic{0.9\linewidth}{../images/great-wave}%
}{%
\blackbf{The great wave off Kanagawa} \\
\textit{Kanagawa-oki nami ura}

\step

\emph{The properties of interest are}
\begin{itemize}
    \item woodblock print from the Edo period
    \item belongs to the Ukiyo-e stream of art
    \item made by Hokusai
\end{itemize}
}

\end{frame}

\begin{frame}

\begin{enumerate}
    \item You can use a table of contents to structure your presentation in \accent{3 - 5} points
    \babystep
    \item The table of content now serves as an \accent{anchor} and helps the audience to follow
    \babystep
    \only<1>{
    \item While optional, the audience should always feel like they know where they are
    }
    \only<2>{
    \item \textbf{Right now, you can transition to your final point}
    }
\end{enumerate}
    
\end{frame}

\begin{frame}
\frametitle{Never show information that you do not yet need \\ by applying the principle of spoon feeding}

\goldencolumn{}{
\onslide<1->
\emph{Only revealing what you currently need} \\
forces the audience to follow you along

\step

\onslide<2->
\emph{Your slides should be understandable on their own} \\
but you guide the audience while presenting

}
    
\end{frame}

\begin{frame}

\begin{enumerate}
    \item You can use a table of contents to structure your presentation in \accent{3 - 5} points
    \babystep
    \only<1>{
    \item The table of content now serves as a \accent{review} and helps the audience to follow
    }
    \only<2->{
    \item \textbf{You need a review now to transition to the conclusion}
    }
    \babystep
    \item While optional, the audience should always feel like they know where they are
\end{enumerate}

\vspace{4em}

\onslide<3>
\goldencolumn{}{
\emph{You can use the table of contents as a review slide} \\
or use a more traditional summary slide
}
    
\end{frame}

\begin{frame}
\frametitle{The conclusion precisely states how you tackle \\ the need from the introduction}

\goldencolumn{
\emph{Your refreshed audience...}

\vfill
}{
\emph{...loves to hear how your method} \\
outperforms SOTA by [quantity] in [metric]

\step

\emph{...can be told that you solved} \\
the most difficult problem in your field

\step

\emph{...is amazed by your quick-witted proof} \\
of a crucial theorem that opens up new perspectives
    
}
    
\end{frame}

\begin{frame}

\attentioncenter{
    You close the presentation by linking your conclusion to the attention-grabbing statement at the start
}
    
\end{frame}

\begin{frame}
\frametitle{Bonus: the general (recursive) structure of a presentation}

\goldencolumn{}{

\colournode{white}{
    \emph{attention-grabber}
}{attention}{0pt} \hfill [audience] \\
\colournode{white}{
    need
}{need}{0pt} \hfill [audience] \\
task \hfill [you] \\
\colournode{white}{
    message
}{message}{0pt} \hfill [vague] \\
preview

\rule[0.5ex]{\linewidth}{1pt}

point 1 \\
transition 1 \\
$\vdots$ \\
transition $N - 1$ \\
point $N$ \hfill [$3 \leq N \leq 5$]

\rule[0.5ex]{\linewidth}{1pt}

review \\
\colournode{white}{
    conclusion
}{conclusion}{0pt} \hfill [precise] \\
\colournode{white}{
    \emph{close}
}{close}{0pt}
}

\begin{tikzpicture}[overlay, remember picture]
\draw[thinpath, shorten <= 1mm, shorten >= 1mm, draw=softest_black] 
    (conclusion.west) -- ([xshift=-1.5em]conclusion.west) |- (need.west);
\draw[thinpath, shorten <= 1mm, shorten >= 1mm, draw=softest_black] 
    (conclusion.west) -- ([xshift=-1.5em]conclusion.west) |- (message.west);
\draw[thinpath, shorten <= 1mm, shorten >= 1mm, draw=softer_black] 
    (close.west) -- ([xshift=-2.5em]close.west) |- (attention.west);
\end{tikzpicture}
\end{frame}

\begin{stopframe}

\goldencolumn{
    \qrcode{../qr/personal_site_qr.pdf}{Visit personal page}%
}{
\emph{This page can display logos, QR codes} \\
or any other content after your presentation

\step

\emph{It can stay open during further Q\&A}
}

\end{stopframe}


\end{document}